\documentclass[a4paper]{article}
\usepackage{amsmath}  % Pacote necessário para \dfrac
\usepackage[utf8]{inputenc}
\usepackage[portuguese]{babel}
\usepackage{graphicx}
\usepackage{array}
\usepackage{fancyhdr}
\usepackage{lastpage} % pacote para obter o número total de páginas
\usepackage{geometry} % pacote para ajustar as margens
\usepackage[datesep=/,style=ddmmyyyy]{datetime2} % pacote para formatar a data
\usepackage{setspace} % Inclui o pacote setspace
\usepackage{enumitem}
\usepackage{amssymb} % Para mais símbolos
\usepackage{indentfirst}
\usepackage{hyperref}
\usepackage{array}
\usepackage{booktabs}
\usepackage{xcolor}
\usepackage{hyperref}
\usepackage{xurl}
\usepackage[style=ieee]{biblatex}
\addbibresource{bibliografia.bib} % nome do seu arquivo .bib
\usepackage{url}

\usepackage{titlesec} % Inclui o pacote titlesec
% Redefinindo o formato do título da seção com um tamanho de fonte menor
\titleformat{\section}
{\normalfont\large\bfseries}{\thesection}{1em}{}

% Definindo margens da página e do cabeçalho
\geometry{
	left=20mm,
	right=20mm,
	top=40mm,
	bottom=30mm,
	headsep=20mm
}

\pagestyle{fancy}
\fancyhf{} % limpa o cabeçalho e rodapé padrão
\renewcommand{\headrulewidth}{0pt} % remove a linha do cabeçalho
\renewcommand{\footrulewidth}{0.4pt} % linha acima do rodapé

\fancyhead[C]{ % conteúdo centralizado no cabeçalho
	\begin{tabular}{|m{3.5cm}|m{9.0cm}|m{3.5cm}|}
		\hline
		\begin{minipage}[c][2.0cm][c]{3.5cm}
			\centering
			\includegraphics[width=2.98cm,height=1.25cm]{logo.png}
		\end{minipage} & 
		\begin{minipage}[c][2.0cm][c]{9cm}
			\centering
			\hyphenpenalty=10000 % Evita hifenização
			\vspace*{\fill} % Espaço vertical flexível antes do texto
			\begin{spacing}{1.5} % Aumenta o espaçamento entre linhas para 1.25
				{\large \textbf{Proteção de Banco de Capacitores}}
			\end{spacing}
			\vspace*{\fill} % Espaço vertical flexível depois do texto
		\end{minipage} & 
		\begin{minipage}[c][2.0cm][c]{3.5cm}
			\raggedleft
			Emissão: \DTMtoday \\
			Folha: \thepage/\pageref{LastPage}
		\end{minipage} \\
		\hline
	\end{tabular}
}


% Conteúdo do rodapé
\fancyfoot[L]{%
	\begin{tabular}[b]{@{}l@{}}
		\href{http://www.dax.energy}{www.dax.energy}
	\end{tabular}
}
\fancyfoot[C]{%
	\begin{tabular}[b]{@{}c@{}}
		\href{mailto:comercial@dax.energy}{comercial@dax.energy}
	\end{tabular}
}
\fancyfoot[R]{%
	\begin{tabular}[b]{@{}r@{}}
		+55 41 99940-3744 \\ 3626-2072
	\end{tabular}
}


\begin{document}
\setstretch{1.25} % Define o espaçamento entre linhas para 1.25

\section{Contexto}
Em sistemas elétricos, especialmente em bancos de capacitores de média tensão, pequenos desequilíbrios são naturais e geralmente esperados durante a operação. Esses desequilíbrios podem ser causados por diversas razões, como diferenças mínimas nas características dos capacitores individuais, variações nas tensões de alimentação, ou condições de carga ligeiramente desiguais. Em níveis baixos, esses desequilíbrios não comprometem a segurança ou a eficiência do sistema e são considerados dentro dos parâmetros normais de operação.

Os bancos de capacitores são projetados para tolerar certos níveis de desequilíbrio sem prejuízos significativos. No entanto, é crucial monitorar continuamente esses desequilíbrios para garantir que permaneçam dentro de limites aceitáveis. Instrumentação adequada e sistemas de monitoramento ajudam a identificar e registrar os níveis de desequilíbrio, permitindo uma gestão proativa da operação do banco de capacitores.

\begin{figure}[htbp]
	\centering
	\includegraphics[width=0.9\linewidth]{"Figure 34 Illustration of an uneven double wye connected bank"}
	\caption{Ilustração de um banco em dupla estrela assimétrica \cite{ieeec3799}.}
	\label{fig:figure-34-illustration-of-an-uneven-double-wye-connected-bank}
\end{figure}

Em uma configuração estrela, o neutro do banco de capacitores não é conectado diretamente à terra. Em condições normais de operação, a tensão $V_{OG}$ (entre neutro e terra) e/ou a corrente $I_n$ (entre estrelas) é muito baixa (idealmente zero). No entanto, no caso de queima parcial de unidades capacitivas, essas grandezas aumentam, indicando a necessidade de uma intervenção, a depender de seu valor.

Para monitorar essas grandezas deslocamento, um Transformador de Potencial (para estrela simples isolada) ou um Transformador de Corrente (dupla estrela isolada) é utilizado para medir a tensão de deslocamento de neutro ou a corrente de desquilíbrio entre as estrelas.  

Por fim, o relé do disjuntor associado ao banco de capacitores comunica-se com o sistema supervisório, permitindo a detecção e o monitoramento em tempo real das condições operacionais e facilitando a tomada de decisão remota em caso de anomalias na tensão de deslocamento do neutro ou corrente de desequilíbrio entre as estrelas.


\section{Níveis de Alerta e de Ação}
Quando os desequilíbrios excedem um nível determinado, mas ainda estão abaixo de um ponto crítico, é necessário atenção redobrada. Esse nível inicial de alerta indica que algo pode estar se desenvolvendo no sistema que requer investigação. Nessa fase, medidas de diagnóstico adicionais devem ser adotadas para identificar a causa do desequilíbrio. Manutenções preventivas, inspeções e ajustes podem ser necessários para corrigir ou mitigar as causas subjacentes.

\section{Nível de Preocupação}
Se o desequilíbrio continuar a aumentar e atinge um nível maior, ele deve ser tratado com preocupação. Nesta fase, o desequilíbrio pode começar a afetar negativamente a eficiência e a segurança do sistema. Os técnicos responsáveis devem realizar uma análise detalhada para determinar se há componentes defeituosos, deterioração dos capacitores, problemas de conexão ou outras falhas que possam estar contribuindo para o desequilíbrio.

\section{Nível crítico - desligamento do banco}
Acima de um certo nível crítico de desequilíbrio, a operação contínua do banco de capacitores pode se tornar perigosa e ineficiente. Esse nível crítico é um ponto de ação definitiva, onde o banco de capacitores deve ser desligado imediatamente para evitar danos aos equipamentos, riscos de segurança e interrupções no fornecimento de energia. O desligamento permite uma inspeção completa e a implementação das correções necessárias antes de o banco de capacitores ser colocado de volta em operação.

\section{TC ou TP de neutro}
A Tabela \ref{tab:df_subset} é uma análise detalhada das condições operacionais de um banco de capacitores, levando em consideração a perda de elementos capacitivos. Esta análise segue as diretrizes estabelecidas pelo \href{https://ieeexplore.ieee.org/document/6466331}{IEEE Std 18 - \textit{IEEE Standard for Shunt Power Capacitors} \cite{ieee18}}, que fornece critérios para proteção e operação de capacitores de potência em sistemas de energia elétrica.

\begin{table}[htbp]
	\centering
	\caption[]{Tabela de desbalanço (Tensões de alarme e desligamento automático do banco de capacitores)}
	\begin{tabular}{ccccccccc}
\toprule
$f$ & $V_e$ [pu] & $V_{cu}$ [pu] & $V_{cu}$ [pu work] & $V_{ln}$ [pu] & $I_u$ [pu] & $V_{ng}$ [pu] & $V_{ng}$ [V] & $I_n$ [A] \\
\midrule
0 & 1.0000 & 1.0000 & 0.7692 & 1.0000 & 1.0000 & 0.0000 & 0.0000 & 0.0000 \\
1 & 1.1185 & 1.0150 & 0.7807 & 1.0012 & 0.9942 & 0.0012 & 22.9213 & 0.2889 \\
\textcolor{blue}{2} & \textcolor{blue}{1.2689} & \textcolor{blue}{1.0339} & \textcolor{blue}{0.7953} & \textcolor{blue}{1.0026} & \textcolor{blue}{0.9869} & \textcolor{blue}{0.0026} & \textcolor{blue}{52.0067} & \textcolor{blue}{0.6554} \\
3 & 1.4661 & 1.0588 & 0.8145 & 1.0045 & 0.9774 & 0.0045 & 90.1293 & 1.1358 \\
\textcolor{yellow!50!black}{4} & \textcolor{yellow!50!black}{1.7357} & \textcolor{yellow!50!black}{1.0929} & \textcolor{yellow!50!black}{0.8407} & \textcolor{yellow!50!black}{1.0071} & \textcolor{yellow!50!black}{0.9643} & \textcolor{yellow!50!black}{0.0071} & \textcolor{yellow!50!black}{142.2756} & \textcolor{yellow!50!black}{1.7930} \\
5 & 2.1269 & 1.1422 & 0.8786 & 1.0109 & 0.9453 & 0.0109 & 217.9276 & 2.7464 \\
6 & 2.7458 & 1.2203 & 0.9387 & 1.0169 & 0.9153 & 0.0169 & 337.6031 & 4.2546 \\
\textcolor{red}{7} & \textcolor{red}{3.8725} & \textcolor{red}{1.3625} & \textcolor{red}{1.0481} & \textcolor{red}{1.0279} & \textcolor{red}{0.8606} & \textcolor{red}{0.0279} & \textcolor{red}{555.4984} & \textcolor{red}{7.0006} \\
8 & 6.5676 & 1.7027 & 1.3098 & 1.0541 & 0.7297 & 0.0541 & 1076.6802 & 13.5687 \\
\bottomrule
\end{tabular}
	\label{tab:df_subset}
\end{table}




Neste caso específico, cada célula capacitiva do banco é composta por 9 elementos em paralelo e 6 elementos em série. Essa configuração garante uma distribuição adequada da tensão e corrente entre os elementos, aumentando a confiabilidade e a capacidade do banco de suportar variações operacionais.

Esta tabela, baseada nas diretrizes do IEEE Std 18, é uma ferramenta valiosa para a operação segura de bancos de capacitores. Ela permite que engenheiros identifiquem rapidamente quando o sistema está se aproximando de condições perigosas, acionando alarmes e trips conforme necessário para proteger os capacitores e o sistema elétrico como um todo. A configuração específica de 9 elementos em paralelo e 6 em série deve ser cuidadosamente monitorada para garantir que a tensão, corrente e capacitância permaneçam dentro dos limites operacionais seguros, prevenindo falhas catastróficas e garantindo a continuidade do serviço.

\printbibliography







% Espaço para assinaturas
\noindent % Evita a indentação
\begin{minipage}[t]{0.5\textwidth} % Inicia a primeira coluna para assinatura
	\centering % Alinha o texto ao centro
	\vspace{5cm} % Espaço reservado para a assinatura
	\rule{6cm}{0.4pt}\\ % Linha para assinatura
	\textbf{Angelo A. Hafner}\\ % Nome
	Engenheiro Eletricista\\ % Título
	CONFEA: 2.500.821.919\\ % Número do registro
	CREA/SC: 045.776-5\\ % Outro número do registro
	aah@dax.energy % E-mail
\end{minipage}%
\hfill % Espaço entre as colunas
\begin{minipage}[t]{0.5\textwidth} % Inicia a segunda coluna para assinatura
	\centering % Alinha o texto ao centro
	\vspace{5cm} % Espaço reservado para a assinatura
	\rule{6cm}{0.4pt}\\ % Linha para assinatura
	\textbf{Tiago Machado}\\ % Nome
	Business Manager\\ % Título
	Mobile: +55 41 99940-3744\\ % Contato
	tm@dax.energy % E-mail
\end{minipage}

\end{document}
